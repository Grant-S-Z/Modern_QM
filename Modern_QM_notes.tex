\documentclass[11pt]{elegantbook}

\title{Modern Quantum Mechanics}
\subtitle{--An abstract}

\author{Grant}
\institute{DEP, THU}
\date{\today}

\cover{R-C(1).jpg}

\definecolor{customcolor}{RGB}{32,178,170}
\colorlet{coverlinecolor}{customcolor}

\begin{document}

\maketitle

\frontmatter
\tableofcontents

\mainmatter

\chapter{Fundamental Concepts}

Our knowledge of the physical world comes from making assumptions about nature, 
formulating these assumptions into postulates, deriving predictions from those 
postulates, and testing those predictions against experiment.

\section{The Stern-Gerlach Experiment}

It is conceived by O. Stern in 1921 and carried out in Frankfurt by him in collaboration 
with W. Gerlach in 1922.

In a certain sense, a two-state system of the Stern-Gerlach type is the least classical, 
most quantum-mechanical system.

\subsection{Description of the Experiment}

We now present a brief discussion of \textbf{the Stern-Gerlach experiment}.

\begin{enumerate}
  \item Silver (Ag) atoms are heated in an oven, which has a small hole through which 
  some of the silver atoms escape.
  \item The beam goes through a collimator and is then subjected to an inhomogeneous 
  magnetic field produced by a pair of pole pieces, one of which has a very sharp edge.
\end{enumerate}

If we ignore the nuclear spin, the atom have an angular momentum of the single 47th (5s) 
electron. As a result, the atom possesses a magnetic moment equal to the spin magnetic moment of the 
47th electron, 
\begin{equation}
  \bm{\mu}\propto \bm{S},
\end{equation}
where the precise propotionality factor turns out to be $e/m_ec$\footnote{$e<0$ in this book} 
to an accuracy of about $0.2\%$.

The z-component of the force is given by 
\begin{equation}
  F_z=\frac{\partial}{\partial z}(\bm{\mu}\cdot \bm{B})\simeq\mu_z\frac{\partial B_z}{\partial z},
\end{equation}
where we have ignored the components in directions other than the z-direction. Obviously, 
the SG (Stren-Gerlach) apparatus "measures" the z-component of $\bm{\mu}$ or, equivalently, 
the z-component of $\bm{S}$ up to a proportionality factor.

The atoms in the oven are randomly oriented. If the electron were like a classical spinning 
object, we would expect all values of $\mu_z$ to be realized between $\vert\bm{\mu}\vert$ and $-\vert\bm{\mu}\vert$. 
This would lead us to expect a continuous bundle of beams coming out of the apparatus.

However, the apparatus splits the original silver beam from the oven into \textit{two distinct} components. 
To the extent that $\bm{\mu}$ can be identified within a proportionality factor with the electron spin 
$\bm{S}$, only two possible values of the z-component of $\bm{S}$ are observed to be possible, which 
we call $S_z+$ and $S_z-$. Numerically it turns out that $\vert S_z\vert=\hbar/2$, where
\begin{equation}
  \hbar=1.0546\times 10^{-27}\rm{erg/s}=6.5822\times 10^{-16}\rm{eV/s}.
\end{equation}
This "quantization" of the electron spin angular momentum is the first important feature 
we deduce from the Stern-Gerlach experiment.

\subsection{Sequential Stern-Gerlach Experiments}

Now consider a \textbf{sequential Stern-Gerlach experiment}.
\begin{enumerate}
  \item Through two SG$\hat{\bm{z}}$ apparatus
  
  Only one beam component coming out of the second apparatus, which is the same as the previous one.
  \item Through SG$\hat{\bm{z}}$ apparatus, and then SG$\hat{\bm{x}}$ apparatus
  
  An $S_x+$ component and an $S_x-$ component coming out with the same intensities.
  \item Through an SG$\hat{\bm{z}}$ and an SG$\hat{\bm{x}}$ apparatus, and then an SG$\hat{\bm{z}}$ apparatus
  
  Both an $S_z+$ component and an $S_z-$ component emerge from the third apparatus.
\end{enumerate}

\textbf{What a Surprise!} This example is often used to illustrate that in quantum mechanics we 
cannot determine both $S_z$ and $S_x$ simultaneously, or the selection of the $S_x+$ beam 
by the second apparatus (SG$\hat{\bm{x}}$) completely destroys any \textit{previous} 
information about $S_z$.

The peculiarities of quantum mechanics are imposed upon us by the experiment itself. 
The limitation is, in fact, inherent in microscopic phenomena.

\subsection{Analogy with Polarization of Light}

We now digress to consider \textbf{the polarization of light waves}.

Consider a monochromatic light wave propagating in the $z$-direction. We can definite an 
\textit{x-polarized light} and \textit{y-polarized light}, 
\begin{align}
  \bm{E}&=E_0\hat{\bm{x}}\cos(kz-\omega t),\\
  \bm{E}&=E_0\hat{\bm{y}}\cos(kz-\omega t).
\end{align}

As for filter, we now consider 2 examples.
\begin{enumerate}
  \item Through an $x$-filter and an $y$-filter 
  
  No light beam comes out.
  \item Insert between the $x$-filter and the $y$-filter another Polaroid - $x'$-direction - 
  that makes an angle of $45^\circ$ with the $x$-direction in the $xy$ plane 

  There is a light beam coming out despite the fact that right after the beam went through the $x$-filter 
  it did not have any polarization component in the $y$-direction.
\end{enumerate}
We'll find these situations are quite analogous to the situations that we encountered 
earlier with the SG arrangement, provided that the following correspondence is made: 
\begin{align}
  \begin{split}
    S_z\pm\rm{atoms}&\leftrightarrow x-,y-\rm{polarized\ light},\\
    S_x\pm\rm{atoms}&\leftrightarrow x'-,y'-\rm{polarized\ light}.
  \end{split}
\end{align}

We know the relation 
\begin{align}
  \begin{split}
    E_0\bm{\hat{x}'}\cos(kz-\omega t)&=E_0[\frac{1}{\sqrt{2}}\bm{\hat{x}}\cos(kz-\omega t)+\frac{1}{\sqrt{2}}\bm{\hat{y}}\cos(kz-\omega t)],\\
    E_0\bm{\hat{y}'}\cos(kz-\omega t)&=E_0[-\frac{1}{\sqrt{2}}\bm{\hat{x}}\cos(kz-\omega t)+\frac{1}{\sqrt{2}}\bm{\hat{y}}\cos(kz-\omega t)].
  \end{split}
\end{align}
In the triple-filter arrangement the beam coming out of the first Polaroid is an $\bm{\hat{x}}$-polarized 
beam, which can be regarded as a linear combination of an $x'$-polarized beam and a 
$y'$-polarized beam. The second Polaroid selects the $x'$-polarized beam, which can in 
turn be regraded as a linear combination of an $x$-polarized and a $y$-polarized beam. 
And finally, the third Polaroid selects the $y$-polarized component.

We might be able to represent the spin state of a silver atom by some kind of vector in 
a new kind of two-dimensional vector space. Just represent the $S_x+$ state by a vector, 
which we call a ket in the Dirac notation. So we conjecture 
\begin{align}
  \begin{split}
    \vert S_x;+\rangle&=\frac{1}{\sqrt{2}}\vert S_z;+\rangle+\frac{1}{\sqrt{2}}\vert S_z;-\rangle\\
    \vert S_z;+\rangle&=-\frac{1}{\sqrt{2}}\vert S_x;+\rangle+\frac{1}{\sqrt{2}}\vert S_x;-\rangle 
  \end{split}
\end{align}
in analogy with (1.7).

Thus the component coming out of the second apparatus is to be regarded as a superposition 
of $S_z+$ and $S_z-$. It is for this reason that two components emerge from the third apparatus.

Another question: \textbf{How to represent the $S_y\pm$ states?}

This time we consider a circularly polarized beam of light, which can be obtained by letting 
a linearly polarized light pass through a quarter-wave plate: 
\begin{equation}
  \bm{E}=E_0[\frac{1}{\sqrt{2}}\bm{\hat{x}}\cos(kz-\omega t)+\frac{1}{\sqrt{2}}\cos(kz-\omega t+\frac{\pi}{2})].
\end{equation}
It is more elegant to use complex notation:
\begin{equation}
  \bm{\epsilon}=[\frac{1}{\sqrt{2}}\bm{\hat{x}}e^{i(kz-\omega t)}+\frac{i}{\sqrt{2}}\bm{\hat{y}}e^{i(kz-\omega t)}],
\end{equation}
where we have used $i=e^{i\pi/2}$.

We can make the following analogy with the spin states of silver atoms:
\begin{align}
  S_y+\rm{atom}&\leftrightarrow \rm{right\ circularly\ polarized\ beam},\\
  S_y-\rm{atom}&\leftrightarrow \rm{left\ circularly\ polarized\ beam}.
\end{align}
Hence, if we are allowed to make the coefficients preceding base kets complex, there is 
no difficulty in accommodating the $S_y\pm$ atoms in our vector space formalism:
\begin{equation}
  \vert S_y;\pm\rangle=\frac{1}{\sqrt{2}}\vert S_z;+\rangle\pm\frac{i}{\sqrt{2}}\vert S_z;-\rangle.
\end{equation}
We thus see that the two-dimensional vector space needed to describe the spin states of 
silver atoms must be a \textit{complex} vector space; an arbitrary vector in the vector 
space is written as a linear combination of the base vectors $\vert S_z;\pm\rangle$ with, 
in general, complex coefficients.

\newpage

\section{Kets, Bras, and Operators}

In this and the following section we formulate the basic mathematics of vector spaces 
as used in quantum mechanics, which was developed by P. A. M. Dirac.

\subsection{Ket Space}

\begin{definition}[Ket]
  A \textbf{state vector} in a complex vector space to represent a physical state in quantum mechanics.
\end{definition}
This state ket is postulated to contain complete information about the physical state; 
everything we are allowed to ask about the state is contained in the ket.

If we multiply ket $\vert\alpha\rangle$ by a complex number $c$, the resulting product 
$c\vert\alpha\rangle$ is another ket. If $c$ is zero, the resulting ket is said to be a 
\textbf{null ket}.

\begin{postulate}
$\vert\alpha\rangle$ and $c\vert\alpha\rangle$, with $c\neq 0$, represent the same physical state.
\end{postulate}
In other words, only the "direction" in vector space is of significance.

\begin{definition}[operator]
  A \textbf{matrix} in the vector space to represent an \textbf{observable}.
\end{definition}
Quite generally, an operator acts on a ket \textit{from the left}.

\begin{definition}[eigenkets, eigenvalues and eigenstate]
  Kets are \textbf{eigenkets} of operator $A$ if $A\vert\alpha\rangle$ is a constant times $\vert\alpha\rangle$, and 
  the constants are called \textbf{eigenvalues} of operator $A$.

  The physical state corresponding to an eigenket is called an \textbf{eigenstate}.
\end{definition}

The dimensionality of the vector space is determined by the number of alternatives in 
Stern-Gerlach type experiments. More formally, we are concerned with an $N$-dimensional 
vector space spanned by the $N$ eigenkets of observable $A$. Any arbitrary ket can be 
wriiten as 
\begin{equation}
  \vert\alpha\rangle=\sum_N^ic_i\vert\alpha_i\rangle.
\end{equation}

\subsection{Bra Space and Inner Products}



\newpage

\section{Base Kets and Matrix Representations}

\newpage

\section{Measurements, Observables, and the Uncertainty Relations}

\newpage

\section{Change of Basis}

\newpage

\section{Position, Momentum, and Translation}

\newpage

\section{Wave Functions in Position and Momentum Space}

\newpage


\nocite{sakurai2020modern}
\printbibliography[heading=bibintoc, title=\ebibname]

\appendix

\chapter{Mathematical Tools}

\end{document}